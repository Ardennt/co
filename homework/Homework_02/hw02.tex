\documentclass[10pt]{article}
\usepackage[margin=1.25in]{geometry}
\usepackage{amsmath, amssymb}

\begin{document}
\begin{center}
    \textbf{CSCI 2500 — Computer Organization\\
    Homework 02 (document version 1.0) — Due September 22, 2022\\
    It’s All About Performance!\\}
\end{center}

\begin{enumerate}
    \item Consider three different processors P1, P2, and P3 executing the same instruction set. P1 has a 3 GHz clock rate and a CPI of 1.5. P2 has a 2.5 GHz clock rate and a CPI of 1.0. P3 has a 4.0 GHz clock rate and has a CPI of 2.2.
    
    \begin{enumerate}
        \item Which processor has the highest performance expressed in instructions per second?
        \begin{center}
            \textbf{P2 has the highest performance}
            \begin{align*}
                P_2 = \frac{2.5 \times 10^9}{1.0} = 2.5 \times 10^9I_s
            \end{align*}

        \end{center}
        
        \item If the processors each execute a program in 10 seconds, find the number of cycles and the number of instructions.
        \begin{center}
            $CPU_t$ = cpu time\\$N_c$ = number of cycles\\$I_c$ = instruction count\\$C_r$ = clock rate\\$t$ = time
            \begin{align*}
                N_c = C_r \times t\\
                I_c = \frac{CPU_t \times C_r}{CPI}
            \end{align*}
        \end{center}
        \begin{enumerate}
            \item P1 \\$N_c = 3 \times 10^9 \times 10 = 3\times10^{10}$ cycles\\$I_c = \frac{3\times10^9\times10}{1.5} = 2\times10^{10}$ \textbf{instructions}
            \item P2 \\$N_c = 2.5 \times 10^9 \times 10 = 2.5 \times 10^{10}$ cycles\\$I_c = \frac{2.5\times10^9\times10}{1.0} = 2.5\times10^{10}$ \textbf{instructions}
            \item P3 \\$N_c = 4 \times 10^9 \times 10 = 4\times10^{10}$ cycles \\ $I_c = \frac{4\times10^9\times10}{2.2} = 1.82\times10^{10}$ \textbf{instructions}
        \end{enumerate}
        

        \item We are trying to reduce the execution time by 30\% but this leads to an increase of 20\% in the CPI. What clock rate should we have to get this time reduction?
        \begin{center}
            $execution_t = \frac{I_c \times CPI}{C_r}$\\
            $execution_t * 0.7 = \frac{I_c \times CPI \times 1.2}{NC_r}$\\
            $NC_r = \frac{I_c \times CPI \times 1.2}{0.7 \times execution_t} =>$ substitute $C_r = \frac{I_c \times CPI}{execution_t}$\\
            $NC_r = \frac{C_r \times 1.2}{0.7}$\\
            $NC_r = 1.71 \times C_r$\\
            Our new clock rate should be \textbf{71\%} greater, or \textbf{1.71} times greater.
        \end{center}
    \end{enumerate}
    \item Consider two different implementations of the same instruction set architecture. The instructions can be divided into four classes according to their CPI (class A, B, C, and D). P1 with a clock rate of 2.5 GHz and CPIs of 1, 2, 3, and 3, and P2 with a clock rate of 3 GHz and CPIs of 2, 2, 2, and 2.

    Given a program with a dynamic instruction count of 1.0E6 instructions divided into classes as follows: 10\% class A, 20\% class B, 50\% class C, and 20\% class D, which implementation is faster?
    \begin{enumerate}
        \item What is the global CPI for each implementation?
        \begin{itemize}
            \item $P_1 = 1\times0.1+2\times0.2+3\times0.5+3\times0.2=\textbf{2.6}$
            \item $P_2 = 2\times0.1+2\times0.2+2\times0.5+2\times0.2=\textbf{2.0}$
        \end{itemize}
        \item Find the clock cycles required in both cases.
        \begin{itemize}
            \item $P_1 = 2.6\times10^6$
            \item $P_2 = 2.0\times10^6$
        \end{itemize}
        \begin{align*}
            E_{t} &= CPI \times I_c \times C_c => C_c = \frac{1}{C_r}\\
            E_{t} &= CPI \times I_c \times C_r^{-1}\\
            E_{t1} &= 2.6 \times 10^6\times (2.5\times10)^{-9} = 1.04 \times 10^{-3}\\
            E_{t2} &= 2.0 \times 10^6\times (3 \times 10)^{-9} = 0.667 \times 10^{-3}
        \end{align*}
        \begin{center}
            P2 is faster due to lower execution time of \textbf{0.667 milliseconds}.
        \end{center}
    \end{enumerate}
    
    \item Compilers can have a profound impact on the performance of an application. Assume that for a program, compiler A results in a dynamic instruction count of 1.0E9 and has an execution time of 1.1 s, while compiler B results in a dynamic instruction count of 1.2E9 and an execution time of 1.5 s.
    \begin{enumerate}
        \item Find the average CPI for each program given that the processor has a clock cycle time of 1 ns.
        \begin{itemize}
            \item $CPU_A = \frac{E_t}{I_c \times C_c} = \frac{1.1}{10^9 \times 10^{-9}} = \textbf{1.1}$
            \item $CPU_a = \frac{E_t}{I_c \times C_c} = \frac{1.5}{1.2\times10^9\times10^{-9}} = \textbf{1.25}$
        \end{itemize}
        \item Assume the compiled programs run on two different processors. If the execution times on the two processors are the same, how much faster is the clock of the processor running compiler A's code versus the clock of the processor running compiler B's code?
        \begin{itemize}
            \item formula for execution time $E_t = CPI \times C_c \times I_c$
            \item from the above, we derive $C_c = \frac{E_t}{CPI \times I_c}$
            \item reciprocate clock cycle for clock rate $C_r = \frac{CPI \times I_c}{E_t}$
            \item we can assume $E_t$ is a constant = 1, therefore we can solve with $\frac{CPI_B \times I_{cB}}{CPI_A \times I_{cA}}$
            \item difference can be calculated with $\frac{1.25 \times 1.2 \times 10^9}{1.1 \times 10^9} = 1.363$
            \item Compiler A is running \textbf{1.363} times faster than Compiler B
        \end{itemize}
        \item A new compiler is developed that uses only 6.0E8 instructions and has an average CPI of 1.1. What is the speedup of using this new compiler versus using compiler A or B on the original processor?
        \begin{itemize}
            \item compiler A: $\frac{1.1 \times 10^9}{1.1 \times 6.0 \times 10^8} = \textbf{1.67}$
            \item compiler B: $\frac{1.25 \times 1.2 \times 10^9}{1.1 \times 6.0 \times 10^8} = \textbf{2.27}$
        \end{itemize}
    \end{enumerate}
    
    \item Assume for arithmetic, load/store, and branch instructions, a processor has CPIs of 1, 12, and 5, respectively. Also assume that on a single processor a program requires the execution of 2.56E9 arithmetic instructions, 1.28E9 load/store instructions, and 256 million branch instructions. Assume that each processor has a 2 GHz clock frequency.

    Assume that, as the program is parallelized to run over multiple cores, the number of arithmetic and load/store instructions per processor is divided by 0.7 x p (where p is the number of processors) but the number of branch instructions per processor remains the same.
    \begin{enumerate}
        \item 
        Find the total execution time for this program on 1, 2, 4, and 8 processors, and show the relative speedup of the 2, 4, and 8 processor result relative to the single processor result.

        \begin{itemize}
            \item Arithmetic Instructions (CPI: 1, Instruction Count: $2.56\times10^9$)
            \item Load/store Instructions (CPI: 12, Instruction Count: $1.28\times10^9$)
            \item Branch Instructions (CPI: 5, Instruction Count: $256\times10^6$)
        \end{itemize}
        \begin{align*}
            E_t &= \frac{\frac{CPI_{ar} \times I_{car}}{0.7p} + \frac{CPI_{ls} \times I_{cls}}{0.7p} + CPI_b \times I_{cb}}{C_r}\\
            &= \frac{\frac{2.56 \times 10^9}{0.7p} + \frac{15.36 \times 10^9}{0.7p} + 1.28 \times 10^9}{2 \times 10^9}\\
            % processor 1
            \text{processor count: 1}\\
            E_t &= \frac{2.56 \times 10^9 + 15.36 \times 10^9 + 1.28 \times 10^9}{2 \times 10^9}\\
            &= \textbf{9.6s}\\
            % processor 2
            \text{processor count: 2}\\
            E_t &= \frac{\frac{2.56 \times 10^9}{0.7(2)} + \frac{15.36 \times 10^9}{0.7(2)} + 1.28 \times 10^9}{2 \times 10^9}\\
            &= \textbf{7.04s}\\
            speedup &= \frac{9.6}{7.04} = \textbf{1.36}\times\\
            % processor 4
            \text{processor count: 4}\\
            E_t &= \frac{\frac{2.56 \times 10^9}{0.7(4)} + \frac{15.36 \times 10^9}{0.7(4)} + 1.28 \times 10^9}{2 \times 10^9}\\
            &= \textbf{3.84s}\\
            speedup &= \frac{9.6}{3.84} = \textbf{2.5}\times\\
            %processor 8
            \text{processor count: 8}\\
            E_t &= \frac{\frac{2.56 \times 10^9}{0.7(8)} + \frac{15.36 \times 10^9}{0.7(8)} + 1.28 \times 10^9}{2 \times 10^9}\\
            &= \textbf{2.24s}\\
            speedup &= \frac{9.6}{2.24} = \textbf{4.29}\times\\
        \end{align*}
        \item
        If the CPI of the arithmetic instructions was doubled, what would the impact be on the execution time of the program on 1, 2, 4, or 8 processors?
        \begin{align*}
            E_t &= \frac{\frac{5.12 \times 10^9}{0.7p} + \frac{15.36 \times 10^9}{0.7p} + 1.28 \times 10^9}{2 \times 10^9}\\
            % processor 1
            \text{processor count: 1}\\
            E_t &= \frac{5.12 \times 10^9 + 15.36 \times 10^9 + 1.28 \times 10^9}{2 \times 10^9}\\
            &= \textbf{10.88s}\\
            % processor 2
            \text{processor count: 2}\\
            E_t &= \frac{\frac{5.12 \times 10^9}{0.7(2)} + \frac{15.36 \times 10^9}{0.7(2)} + 1.28 \times 10^9}{2 \times 10^9}\\
            &= \textbf{7.95s}\\
            % processor 4
            \text{processor count: 4}\\
            E_t &= \frac{\frac{5.12 \times 10^9}{0.7(4)} + \frac{15.36 \times 10^9}{0.7(4)} + 1.28 \times 10^9}{2 \times 10^9}\\
            &= \textbf{4.30s}\\
            % processor 8
            \text{processor count: 8}\\
            E_t &= \frac{\frac{5.12 \times 10^9}{0.7(8)} + \frac{15.36 \times 10^9}{0.7(8)} + 1.28 \times 10^9}{2 \times 10^9}\\
            &= \textbf{2.47s}\\
        \end{align*}
        \item
        To what should the CPI of load/store instructions be reduced in order for a single processor to match the performance of four processors using the original CPI values?
        \begin{align*}
            E_{t4} &= \frac{CPI_{ar1} \times I_{car1} + CPI_{ls1} \times I_{cls1} + CPI_{b1} \times I_{cb1}}{C_r}\\
            3.84 &= \frac{2.56 \times 10^9 +  CPI_{ls1} \times 1.28 \times 10^9 + 1.28 \times 10^9}{2 \times 10^9}\\
            CPI_{ls1} &= \textbf{3.01}
        \end{align*}
    \end{enumerate}
    \item The results of the SPEC CPU2006 bzip2 benchmark running on an AMD Barcelona has an instruction count of 2.389E12, an execution time of 750 s, and a reference time of 9650 s.
    \begin{enumerate}
        \item
        Find the CPI if the clock cycle time is 0.333 ns.
        \begin{align*}
            E_t &= C_c \times CPI \times I_c\\
            CPI &= \frac{E_t}{C_c \times I_c}\\
            CPI &= \frac{750}{0.333 \times 10^{-9} (2.389 \times 10^{12})}\\
            &= \frac{750}{795.537}\\
            &= \textbf{0.942}
        \end{align*}
        
        \item
        Find the SPECratio.
        \begin{align*}
            SPEC_r = \frac{R_t}{E_t} &= \frac{9650}{750}\\
            &= \textbf{12.87}
        \end{align*}
        
        \item
        Find the increase in CPU time if the number of instructions of the benchmark is increased by 10\% without affecting the CPI.
        \begin{align*}
            E_t\ new = E_t \times 1.1 \rightarrow E_t\ new &= 750 \times 1.1\\
            &= \textbf{825}
        \end{align*}
        \item
        Find the increase in CPU time if the number of instructions of the benchmark is increased by 10\% and the CPI is increased by 5\%.
        \begin{align*}
            E_t\ new = E_t \times (1.1 \times 1.05) \rightarrow E_t\ new &= 750 \times (1.155)\\
            &= \textbf{866.25}
        \end{align*}
        \item
        Find the change in the SPECratio for this change.
        \begin{align*}
            \text{Change in spec ratio for (c)}\\
            SPEC_r\ c = \frac{9650}{825} &= 11.697\\
            \frac{original\ spec}{current\ spec} = \frac{12.87}{11.70} &= 1.1\\
            \text{The original SPECratio was \textbf{1.1} times greater}
        \end{align*}
        \begin{align*}
            \text{Change in spec ratio for (d)}\\
            SPEC_r\ c = \frac{9650}{866.25} &= 11.14\\
            \frac{original\ spec}{current\ spec} = \frac{12.87}{11.14} &= 1.2\\
            \text{The original SPECratio was \textbf{1.2} times greater}
        \end{align*}
        
        \item
        Suppose that we are developing a new version of the AMD Barcelona processor with a 4 GHz clock rate. We have added some additional instructions to the instruction set in such a way that the number of instructions has been reduced by 15\%. The execution time is reduced to 700 s and the new SPECratio is 13.7. Find the new CPI.
        \begin{align*}
            C_r &= 4 \times 10^9\\
            new\ I_c = 0.85I_c &= 2.03\\
            E_t &= 700\\
            SPEC_r &= 13.7\\\\
            CPI = \frac{E_t \times C_r}{I_c} &= \frac{700 \times 4 \times 10^9}{2.03 \times 10^{12}}\\
            &= 1.38\\
            \text{The new CPI is \textbf{1.38}}
        \end{align*}
        \item
        This CPI value is larger than obtained in 1.11.1 as the clock rate was increased from 3 GHz to 4 GHz. Determine whether the increase in the CPI is similar to that of the clock rate. If they are dissimilar, why?
        \begin{itemize}
            \item CPI and Clock Rate aren't the only variables that have been changed
            \item If we divide CPI old by new, $\frac{CPI_o}{CPI_n} = \frac{1.38}{0.942} = 1.46$
            \item CPI change was \textbf{0.46}
            \item Clock rate change was $\frac{C_{c\ new}}{C_{c\ old}} = \frac{4 \times 10^9}{3 \times 10^9} = 1.33$
            \item Clcok rate change was \textbf{0.33}
            \item They are dissimilar because other variables uch as execution time and the number of instructions were also changed.
        \end{itemize}
        \item
        By how much has the CPU time been reduced?
        \begin{itemize}
            \item CPU time has been reduced by $\frac{750}{700} = 1.07$, it has been reduced by \textbf{7\%}
        \end{itemize}
        \item
        For a second benchmark, libquantum, assume an execution time of 960 ns, CPI of 1.61, and clock rate of 3 GHz. If the execution time is reduced by an additional 10\% without affecting to the CPI and with a clock rate of 4 GHz, determine the number of instructions.
        \begin{align*}
            E_t = 960 \times 10^{-9} \rightarrow E_t = 960 \times 0.9 \times 10^{-9} &= 864 \times 10^{-9}\\
            CPI &= 1.61\\
            C_r = 3 \times 10^9 -> C_{r\ new} &= 4 \times 10^9\\
            I_c = \frac{E_t \times C_r}{CPI} &= \frac{864 \times 10^{-9} \times 4 \times 10^9}{1.61}\\
            &= 2146.6 \textbf{ instructions}
        \end{align*}
        \item
        Determine the clock rate required to give a further 10\% reduction in CPU time while maintaining the number of instructions and with the CPI unchanged.
        \begin{align*}
            E_t &= 864 \times 0.9 \times 10^{-9}\\
            C_r = \frac{I_c \times CPI}{E_t} &= \frac{2146.6 \times 1.61}{864 \times 0.9 \times 10^{-9}}\\
            &= 4.4 \times 10^9 \leftarrow \textbf{clock rate necessary}
        \end{align*}
        \item
        Determine the clock rate if the CPI is reduced by 15\% and the CPU time by 20\% while the number of instructions is unchanged.
        \begin{align*}
            CPI &= 0.85 \times 1.61\\
            E_t &= 0.8 \times 960 \times 10^{-9}\\
            I_c &= 2146.6\\
            C_r = \frac{I_c \times CPI}{E_t} &= \frac{2146.6 \times 0.85 \times 1.61}{0.8 \times 960 \times 10^{-9}}\\
            &= 3.8 \times 10^9 \leftarrow \textbf{is the clock rate given the parameters}
        \end{align*}
    \end{enumerate}
    \item Assume a program requires the execution of $50 \times 10^6$ FP instructions, $110 \times 10^6$ INT instructions, $80 \times 10^6$ L/S instructions, and $16 \times 10^6$ branch instructions. The CPI for each type of instruction is 1, 1, 4, and 2, respectively. Assume that the processor has a 2 GHz clock rate.
    \begin{enumerate}
        \item 
        By how much must we improve the CPI of FP instructions if we want the program to run two times faster?
        \begin{align*}
            E_t &= \frac{CPI \times I_c}{C_r}\\
            &= \frac{(50 \times 10^6) + (110 \times 10^6) + (80 \times 10^6 \times 4) + (16 \times 10^6 \times 2)}{2 \times 10^9} = 0.256\\
            &\text{To calculate half the time, we have $E_{tnew} = \frac{E_t}{2}$}\\
            0.128 &= \frac{(50 \times 10^6 \times CPI_{FP}) + (110 \times 10^6) + (80 \times 10^6 \times 4) + (16 \times 10^6 \times 2)}{2 \times 10^9}\\
            &\text{Rearrange variables}\\
            (50 \times 10^6) CPI_{FP} &= 0.256 \times 10^9 - 462 \times 10^6\\
            &CPI_{FP}= \frac{0.256 \times 10^9 - 0.462 \times 10^9}{50 \times 10^6}\\
            &= -4.12 \textbf{ Not Applicable}
        \end{align*}
        \item
        By how much must we improve the CPI of L/S instructions if we want the program to run two times faster?
        \begin{align*}
            0.128 &= \frac{(50 \times 10^6) + (110 \times 10^6) + (80 \times 10^6 \times CPI_{L/S}) + (16 \times 10^6 \times 2)}{2 \times 10^9}\\
            64 \times 10^6 &= 80 \times 10^6 \times CPL_{L/S}\\
            CPL_{L/S} &= \frac{64 \times 10^6}{80 \times 10^6}\\
            &= 0.8
            \text{$CPL_{L/S}$ before was 4, new is 0.8}\\
            \frac{4}{0.8} &= 5 \textbf{ CPI must improve by 5 times}
        \end{align*}
        \item
        By how much is the execution time of the program improved if the CPI of INT and FP instructions is reduced by 40\% and the CPI of L/S and Branch is reduced by 30\%?
        \begin{align*}
            E_t &= \frac{(50 \times 10^6)(0.6) + (110 \times 10^6)(0.6) + (80 \times 10^6)(4 \times 0.7) + (16 \times 10^6)(2 \times 0.7)}{2 \times 10^9}\\
            &= 0.1712 -> \frac{0.256}{0.1712} = 1.495 -> \textbf{Execution Time Improved by 1.495}\\
        \end{align*}
    \end{enumerate}
\end{enumerate}

\end{document}