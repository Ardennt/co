\documentclass{article}
\usepackage[margin=1.25in]{geometry}
\usepackage{amsmath, amssymb, setspace, enumerate, enumitem}
\usepackage{setspace}
\usepackage{graphicx}
\doublespacing

\author{Jericho Dizon\\Jose Idrovo\\Adrian Rodriguez\\Jiawei Wu}
\title{\textbf{MIPS Datapath Group Project}}
\begin{document}
     \begin{titlepage}
        \maketitle
     \end{titlepage}

     \begin{enumerate}
        \setcounter{enumi}{1}
        \item \textbf{Verilog Model}
        \\ The Verilog code we selected is from Zybooks 12.9.1, it is attached in file verilog.v
        \item \textbf{Datapath}
        \\ We implemented a pipelined datatpath in our verilog code.
        \item \textbf{Memory}
        \\ Single level memory implementation
        \item \textbf{Simulation}
        \\ Sample MIPS program 
        \begin{verbatim}
add $s1,$s2,$s3
add $s4,$s1,$s3
j dest

Contents of each instruction
r-type instructions
line    opcode[31:26]   rs[25:11]   rt[20:16]   rd[15:11]   shAmt[10:6] funct[5:0]
1       000000          10010 (s2)  10011 (s3)  10001 (s1)  X           100000
2       000000          10001 (s1)  10011 (s3)  10100 (s4)  X           100000

i-type instruction
line    opcode[31:26]   address[25:0]
3       000010          dest
        \end{verbatim}

        Verilog Processor Execution: ()
        \textbf{Cycle 1 (Instruction Fetch):}
        \\ In cycle 1 of execution, PC, Instruction, and the Add ALU is used
        \\ \textbf{PC: } Input: 0, Output = 0
        \\ \textbf{Instruction Memory: } Input: 0, 

        \textbf{Cycle 2 (Instruction Decode):}
        \\ In cycle 2 of execution, WriteReg, our Control Unit, Register File, and Sign Extend are used. 
        \\ \textbf{Execution}

        \textbf{Cycle 3 (Execution):}

        \textbf{Cycle 4 (Memory):}

        \textbf{Cycle 5 (Write Back):}

        \item \textbf{References}
        \\ https://bellerofonte.diism.unisi.it/index.php
        \\ Computer Organization and Design - Interative Version (MIPS)
        \\ FOR REFERENCE AMIGOS:
        \\  Single-level memory is organizing memory with processor and (slow) main memory (DRAM)
        \\  Compare with multilevel memory, which has at least one (quick) cache between the processor and main memory.
        
        % MULT4to1 AND MIPS ALU %
        %https://learn.zybooks.com/zybook/RPICSCI2500KuzminFall2022/chapter/9/section/2?content_resource_id=61945269%

        \item \textbf{Contributions}
        \\  Jose Idrovo:
        \\  
        \\  Adrian Rodriguez:
        \\  
        \\  Jericho Dizon:
        \\  
        \\  Jiawei Wu: 

     \end{enumerate}

     
     
     
\end{document}